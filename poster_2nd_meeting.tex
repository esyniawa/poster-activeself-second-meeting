\documentclass[portrait,final,a0paper,fontscale=0.33]{baposter}

%% read in constants, custom functions and used packages
\input{functions/packages}

\begin{document}

\begin{poster}%
	% Poster Options
	{
		% Show grid to help with alignment
		grid=false,
		% Number of columns and column spacing
		columns=6,
		colspacing=1em,
		% Color style
		bgColorOne=white,
		borderColor=iftuccolor,
		headerColorOne=iftucbackground,
		headerFontColor=iftucfont,
		boxColorOne=white,
		% Format of textbox
		textborder=rounded,
		textfont=\small,
		% Format of text header
		eyecatcher=true,
		headerborder=closed,
		headerheight=0.1\textheight,
		%  textfont=\sc, An example of changing the text font
		headershape=rounded,
		headershade=plain,
		headerfont=\Large\bf, %Sans Serif
		% textfont={\setlength{\parindent}{1.5em}},
		boxshade=plain,
		%  background=shade-tb,
		background=plain,
		linewidth=2pt
	}
	% University logo
	{\includegraphics[height=6.5em]{tuckhseng_color}} 
	% Title
	{\bf\Large{Developing a body schema by multi-sensory\\ integration through recurrent basis functions and the contribution of the basal ganglia to motor learning}\vspace{5pt}}
	% Authors
	{\large Erik~Syniawa, and Fred~Hamker \\ \vspace{0.5em}
	\small Contact: erik.syniawa@informatik.tu-chemnitz.de
	}
	% Department logo and other logos
	{	
		\begin{minipage}[r]{0.1\textwidth}
			\includegraphics[height=7em]{active_self_logo_color}
		\end{minipage}
		\hfill
		\begin{minipage}[r]{0.1\textwidth}
			\includegraphics[height=6.5em]{TUC_AI_color}
		\end{minipage}
		
	}

%%%%%%%%%%%%%%%%%%%%%%%%%%%%%%%%%%%%%%%%%%%%%%%%%%%%%%%%%%%%%%%%%%
% use height in headerbox to align multiple boxes 
% height= <size in percent of column height>, else [auto]%
\headerbox{Overview}{name=overview,column=0,row=0, span=3}{
	
	
	\begin{adjustbox}{minipage=0.95\textwidth, margin=5pt, center}

	\begin{minipage}[l]{\textwidth}
		\justifying
		\textbf{Main question:} \\		
		How can a robot develop \textit{awareness} of its own body by associating proprioception with touch and vision using sensory consequences of motor action? \\
		
		\textbf{Neuro-computational model:} \\
		Our model links \textit{sensory representations} within an integrated \textit{body schema}. Predictions and actual sensory results will be considered in the \textit{basal ganglia}. Through cortico-basal ganglia-thalamo-cortical loops the signal transmission will be modulated and dynamically influence the body schema. 

	\end{minipage}

	\end{adjustbox}
}

\headerbox{\large Sensory Integration\textsuperscript{1}}{name=rbf, column=0, below=overview, span=2, height=0.302}{
	\begin{adjustbox}{minipage=0.95\textwidth, margin=5pt, center}

	\end{adjustbox}
}

\headerbox{\large Learning a Body Schema\textsuperscript{1}}{name=network, column=2, below=overview, span=4, height=0.302}{
	\begin{adjustbox}{minipage=0.95\textwidth, margin=5pt, center}
	% Network definitions

	\end{adjustbox}
}

\headerbox{\large References}{name=refs, column=0, above=bottom, span=6}{
	\begin{adjustbox}{minipage=0.98\textwidth, margin=0pt, center}
		
		\compressbib{\printbibliography[heading=none]}
		
		
	\end{adjustbox}
	
	
}


\headerbox{\Large Setup}{name=setup,column=3,row=0, span=3, above=network}{
	\begin{adjustbox}{minipage=0.95\textwidth, margin=5pt, center}
		\begin{minipage}[l]{0.5\textwidth}
		\vspace{1pt}
		
		\centering
		\includegraphics[width=0.75\linewidth]{robot_setup}
		\captionof{figure}{Current virtual robot setup.}
		\end{minipage}
		\begin{minipage}[l]{0.5\textwidth}
			\textbf{Task:} \\
			Based in 2D-plane a robot should touch this right forearm. He develops the plan and visually fixates the correspondes point on this forearm. 
		\end{minipage}
	\end{adjustbox}
}

\headerbox{\large Synaptic plasticity in the
	Basal Ganglia\textsuperscript{2}}{name=bg, column=0, below=rbf, above=refs , span=3}{
	\begin{adjustbox}{minipage=0.95\textwidth, margin=5pt, center}
		
	
		\begin{minipage}[l]{0.45\textwidth}
			\textbf{Network of the Basal Ganglia (BG):}\\
			
			\vspace{15pt}
			\begin{flushright}
				\includegraphics[width=\linewidth]{BG}
				\captionof{figure}{Modeling of segregated \\ basal ganglia pathways}
			\end{flushright}

			
			\vspace{25pt}
			\justifying
			
			Through \textit{dopamine-modulated plasticity}, the BG enable motor category learning \parencite{segerHowBasalGanglia2008a} and are involved in establishing associations between stimulus and responses \parencite{packardLearningMemoryFunctions2002a}. They act as a kind of reinforcement learning agent. \\
			In our model the BG consist of 3 different pathways. All of them represent actual connections between the different nuclei of the BG (see Figure 6).
			
			\vspace{5pt}
			
		\end{minipage}
		\hfill
		\begin{minipage}[r]{0.55\textwidth}
			\textbf{Learning in the different pathways:}\\
			\justifying
			
			The learning principles are primarily determined by \textit{presynaptic} and \textit{postsynaptic} activity, as well as the \textit{Dopamine signal} (\textbf{DA}). Together these principles form a 3-factor learning rule (see Table 1, modified after \cite{maithOptimalAttentionTuning2021c}).
			\vspace{1pt}
			\begin{itemize}
				\item \textit{High} and \textit{low} indicate whether the pre- and post-activity is more than or less than a given threshold (e.g. mean activity).
				\item \textit{DA+} and \textit{DA-} labels indicate if the DA levels exceed a given threshold or not.
				\item The sign \textit{+} or \textit{-} represents the weight changes in the relevant projections for each combination.
			\end{itemize}
			
			
			\resizebox{\columnwidth}{!}{%
	\begin{tabular}{lcccccl}
		\multicolumn{2}{l}{\multirow{4}{*}{}}          & \multicolumn{4}{c}{\textbf{Dopamine}}               & \multirow{4}{*}{}                    \\ \cline{3-6}
		\multicolumn{2}{l}{}                           & \multicolumn{2}{c}{DA +} & \multicolumn{2}{c}{DA -} &                                      \\ \cline{3-6}
		\multicolumn{2}{l}{}                           & \multicolumn{4}{c}{\textbf{Post-activity}}          &                                      \\ \cline{3-6}
		\multicolumn{2}{l}{}                           & High        & Low        & High        & Low        &                                      \\ \hline
		\multirow{8}{*}{\textbf{Pre-activity}} & High & +           &            & -           &            & \multirow{2}{*}{\textbf{Cortex-D1}}  \\
		& Low  & -           &            &             &            &                                      \\ \cline{2-7} 
		& High & -           &            & +           &            & \multirow{2}{*}{\textbf{Cortex-D2}}  \\
		& Low  &             &            & -           &            &                                      \\ \cline{2-7} 
		& High & -           & +          &             & -          & \multirow{2}{*}{\textbf{D1-GPi}}     \\
		& Low  &             &            &             &            &                                      \\ \cline{2-7} 
		& High &             & -          & -           & +          & \multirow{2}{*}{\textbf{D2-GPe}}     \\
		& Low  &             &            &             &            &                                      \\ \hline
	\end{tabular}
}
			\captionof{table}{"+"=LTP; "-"=LTD; no sign = no weight change}
		\end{minipage}
	\end{adjustbox}

}

\headerbox{\large Motor Learning in the Basal Ganglia\textsuperscript{2}}{name=motor, column=3, below=network, above=refs , span=3}{

	\begin{adjustbox}{minipage=0.95\textwidth, margin=5pt, center}
		
	\end{adjustbox}

}


\end{poster}


\end{document}